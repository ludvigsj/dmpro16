\documentclass[10pt]{beamer}
\usetheme[
%%% option passed to the outer theme
%    progressstyle=fixedCircCnt,   % fixedCircCnt, movingCircCnt (moving is deault)
  ]{Feather}

% If you want to change the colors of the various elements in the theme, edit and uncomment the following lines

% Change the bar colors:
%\setbeamercolor{Feather}{fg=red!20,bg=red}

% Change the color of the structural elements:
%\setbeamercolor{structure}{fg=red}

% Change the frame title text color:
%\setbeamercolor{frametitle}{fg=blue}

% Change the normal text color background:
%\setbeamercolor{normal text}{fg=black,bg=gray!10}

%-------------------------------------------------------
% INCLUDE PACKAGES
%-------------------------------------------------------

\usepackage[utf8]{inputenc}
\usepackage[english]{babel}
\usepackage[T1]{fontenc}
\usepackage{helvet}

%-------------------------------------------------------
% DEFFINING AND REDEFINING COMMANDS
%-------------------------------------------------------
% colored hyperlinks
\newcommand{\chref}[2]{
  \href{#1}{{\usebeamercolor[bg]{Feather}#2}}
}
%-------------------------------------------------------
% INFORMATION IN THE TITLE PAGE
%-------------------------------------------------------

\title[] % [] is optional - is placed on the bottom of the sidebar on every slide
{ % is placed on the title page
      \textbf{Sudo-ku}
}

\subtitle[DM Pro 2016]
{
%      \textbf{v. 1.0.0}
}

\author[Energieffektivitetsgruppen]
{      Energieffektivgruppen \\
 %     {\ttfamily dmpro16@samfundet.no}
}

\institute[]
{
%      Faculty of Mathematics, Informatics and Information Technologies\\
%      Plovdiv University ``Paisii Hilendarski''\\

  %there must be an empty line above this line - otherwise some unwanted space is added between the university and the country (I do not know why;( )
}

\date{\today}

%-------------------------------------------------------
% THE BODY OF THE PRESENTATION
%-------------------------------------------------------

\begin{document}

%-------------------------------------------------------
% THE TITLEPAGE
%-------------------------------------------------------

{\1% % this is the name of the PDF file for the background
\begin{frame}[plain,noframenumbering] % the plain option removes the header from the title page, noframenumbering removes the numbering of this frame only
  \titlepage % call the title page information from above
\end{frame}}


\begin{frame}{Content}{}
\tableofcontents
\end{frame}

%-------------------------------------------------------
\section{Planned Operation}
%-------------------------------------------------------
\begin{frame}{Planned operation of sudo-ku}
\begin{itemize}
\item Point camera at an image of a solved sudoku-board
\item Push button
\item Camera image is stored in FPGA BRAM
\item FPGA figures out the transform required to rotate, scale and crop the input image
\item FPGA performs transform, cutting it into pieces, trimming edges of the squares to avoid sending outlines into the BNN etc.
\item BNN (on FPGA) detects digits
\item Digits are sent to the MCU, where the sudoku is checked and the result is output to the user
\end{itemize}
\end{frame}

\begin{frame}{Planned operation of sudo-ku}
\centering{
	\includegraphics[height=.7\textheight]{graphics/lifetime_old.pdf}
}
\end{frame}

\begin{frame}{Planned operation of sudo-ku}{Picture requirements}
\begin{itemize}
\item All but the board itself should be black, the board itself white
\item The camera must be pointed roughly orthogonally to the board; we do not do any perspective transformations
\item The camera can be rotated along the camera-pointing axis by 45 degree either way. This requirement might go away if we find a simple way to figure out the orientation without it.
\end{itemize}
\end{frame}

%-------------------------------------------------------
\section{Design Overview}
%-------------------------------------------------------
\begin{frame}{Design Overview}
\centering{
	\includegraphics[height=.7\textheight]{graphics/tegning.pdf}
}
\end{frame}


%-------------------------------------------------------
\section{Argumentation}
%-------------------------------------------------------
\begin{frame}{Argumentation}{Camera (and SD-card) connected to FPGA}
\begin{itemize}
\item We need 9x9 times 28x28 pixels of relevant image data. Results in 9x9x28x28x2 = 127008 pixels.
\item Smallest standard accommodating this is HVGA(480x320=153600 pixels). But is uncommonly supported. Hence we use VGA(640x480=307200 pixels).
\item We use 1 bit black/white to store the images. Thus we need 307200 bitsof storage.
\item Image from the the camera is in a different format than our 1-bit black/white. This needs conversion on the fly.
\item Considering the size of the image, it is possible to do keep the image in the memory of the FPGA without the use of an external SRAM
\end{itemize}
\end{frame}

\begin{frame}{Argumentation}{Image transformation on FPGA}
\begin{itemize}
\item With the restrictions we put on the picture we can find the corners with a somewhat naive algorithm that should be possible to implement on the FPGA
\item The process of doing an affine transformation on the camera input data is a simple, arithmetical operation, easy to implement elegantly on an FPGA.
\item The resulting, transformed image/images are only needed on the FPGA
\end{itemize}
\end{frame}

\begin{frame}{Argumentation}{User I/O on MCU}
\begin{itemize}
\item Output, which includes communicating with a display, is a lot more difficult implementing in hardware, with no real benefit.
\item The MCU is used to control the operation flow of \textit{sudo-ku}.
\end{itemize}
\end{frame}

\begin{frame}{Argumentation}{Sudoku-checking on MCU}
\begin{itemize}
\item Super easy to do on MCU.
\item Have already made a first draft of about 50 lines of code checking a sudoku.
\item Lots of eye-candy can be implemented when everything works
\item Only ~700 ways to solve a sudoku. The entire board can be transferred from the FPGA to the MCU in 10 bits. Data transferred is negligible.
\end{itemize}
\end{frame}

\begin{frame}{Argumentation}{Why we need SD-card}
\begin{itemize}
\item In case the camera processing does not work, we will still have a way to demo the working BNN by loading pre-formatted, square image data, skipping the transformation steps.
\end{itemize}
\end{frame}

%-------------------------------------------------------
\section{PCB Design}
%-------------------------------------------------------

\begin{frame}{PCB Design}{Top level schematic}
\centering{
	\includegraphics[height=.7\textheight]{graphics/pcb-top-level.png}
}
\end{frame}

\begin{frame}{PCB Design}{3D Drawing}
\centering{
	\includegraphics[height=.7\textheight]{graphics/pcb-3d.png}
}
\end{frame}

%-------------------------------------------------------
\section{Data flow and memory}
%-------------------------------------------------------

\begin{frame}{Data flow and memory}{On the FPGA}
\centering{
	\includegraphics[height=.7\textheight]{graphics/mem_top.pdf}
}
\end{frame}

\begin{frame}{Data flow and memory}{In the Transformer-module}
\centering{
	\includegraphics[height=.7\textheight]{graphics/mem_trans.pdf}
}
\end{frame}

%-------------------------------------------------------
\section{Image transformation}
%-------------------------------------------------------

\begin{frame}{Image transformation}{Cropping and straightening}
\begin{itemize}
    \item Done by a matrix transform \emph{from} the required destination image coordinates (pixels) \emph{to} input-image coordinates (sub-pixel)
    \item Uses coordinates of the topmost corners
\end{itemize}
\pause

    \begin{equation*}
        T(x,y) = \begin{bmatrix}
                a&-b \\
                b&a
            \end{bmatrix} \cdot
            \begin{bmatrix}x\\y\end{bmatrix}
                + \begin{bmatrix}x_0\\y_0\end{bmatrix}
    \end{equation*}
    \begin{align*}
        a &= \frac{x_1 - x_0}{w_\text{dest}} &
        b &= \frac{y_1 - y_0}{w_\text{dest}}
    \end{align*}
    % Where $x_0, y_0$ and $x_1, y_1$ are the top-left and top-right corners, and $w_\text{dest}$ is the width/height of the destination image.
\end{frame}

\begin{frame}{Image transformation}{Slicing and padding}
    \begin{itemize}
        \item A \emph{slice} is a small square containing one digit
        \item The pixels are to be sent into the BNN slicewise
    \pause
        \item Going from slice coordinates $(n, x', y')$, $n$ being the slice number, to sudoku-wide coordinates $(x,y)$ with padding $w_p$:
    \end{itemize}
    \begin{align*}
        x &= (28+2w_p)(n\bmod 9) + x' + w_p \\
        y &= (28+2w_p)\lfloor n/9\rfloor + y' + w_p
    \end{align*}
    \pause
    \begin{itemize}
        \item This makes $w_\text{dest} = 28\cdot 9 + 2\cdot 9\cdot w_p = 252+18w_p$
    \end{itemize}
\end{frame}
%-------------------------------------------------------
\section{Binarizing the neural network}
%-------------------------------------------------------
\begin{frame}{Binarizing the neural network}{Changing to -1 and 1}
\begin{itemize}
	\item We can represent most of the neural network as -1 or 1 without much accuracy loss
	\item The input 0-255 can be -1 if $\leq$ 127 else 1
	\item New weights become sign(old weights)
	\item The activation function tanh(x) can also be replaced by sign(x)
	\item Now only the batch normalization is not represented by -1 and 1 yet
\end{itemize}
\end{frame}

\begin{frame}{Binarizing the neural network}{From multiplication to XNOR}
	\begin{columns}[T] % align columns
		\begin{column}{.48\textwidth}

			\begin{tabular}{r|c|c|}
				  & -1 & 1 \\ \hline
				1 & -1 & 1 \\ \hline
			   -1 &  1 & -1 \\
				\hline
			\end{tabular}

			Multiplication
		\end{column}%
		\hfill%
		\begin{column}{.48\textwidth}
			\begin{tabular}{r|c|c|}
				  & 0 & 1 \\ \hline
				1 & 0 & 1 \\ \hline
			    0 & 1 & 0 \\ \hline
			\end{tabular}

			XNOR
		\end{column}%
	\end{columns}
\end{frame}

\begin{frame}{Binarizing the neural network}{Finding the batch normalization threshold}
\begin{itemize}
	\item $y = \gamma \sigma^{-1}(x-\bar{x})+\beta$
	\item $y = \gamma \sigma^{-1}(2x-\bar{x}-prev)+\beta$
	\item Find where the output is 0, so we can use only one value
	\item $x = 0.5(\bar{x}-\frac{\beta}{\gamma \sigma^{-1}}+prev) $
\end{itemize}
\end{frame}

\begin{frame}{Binarizing the neural network}{The binarized result}
\begin{itemize}
	\item $\sum w \otimes out$
	\item 0 if the sum is below the batchnorm threshold
	\item 1 otherwise
\end{itemize}
\end{frame}

\begin{frame}{BNN Architecture}{Image of the architecture}
\centering{
	\includegraphics[width=\textheight]{graphics/bnn.png}
}
\end{frame}

\begin{frame}{BNN Architecture}{Pipelining}
\begin{itemize}
    \item The next neuron in one layer can start processing one clock cycle after the current neuron
    \item We can start on the next image as soon as the previous one is done with the first layer
    \item All layers can be used at the same time, each working on a different image
\end{itemize}
\end{frame}

\end{document}
